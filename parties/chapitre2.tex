%%- Deuxiemme chapitre de démonstration -%%
\chapter{Ajout d'un second chapitre}

\section{Test de mise en page d'un tableau}

Les tableaux sont soumis aux mêmes contraintes que les figures, en dehors de la position de la légende qui doit être au-dessus, exemple Tableau \ref{chap2:tableau1}.

\begin{table}
    \parbox{0.65\textwidth}{%
    \caption{Test de longue légende pour un tableau, avec retour à la ligne.}%
    \label{chap2:tableau1}%
    } % Contrainte manuelle de la largeur de la légende
    \begin{tabular}{|c|c|c|c|c|c|c|c|}
	\hline
		      {\bf titre} & {\bf titre} & {\bf titre} & {\bf titre} & {\bf titre} & {\bf titre} & {\bf titre} & {\bf titre} \\
	  \hline
			blá & blá & blá & blá & blá & blá & blá & blá \\
	  \hline
			blá & blá & blá & blá & blá & blá & blá & blá \\
	  \hline
			blá & blá & blá & blá & blá & blá & blá & blá \\
	  \hline
			blá & blá & blá & blá & blá & blá & blá & blá \\
	  \hline
			blá & blá & blá & blá & blá & blá & blá & blá \\
	  \hline
			blá & blá & blá & blá & blá & blá & blá & blá \\
	  \hline
    \end{tabular}
\end{table}


\section{Test des références}

\subsection{Références à la bibliographie}

Citation d'une référence de la bibliographie \citep{BookExample}.

\subsection{Références à la liste de références "refs"}

Citation d'une référence de la liste de référence "refs" déclarée au début du document \citerefs{classURL}.

\subsection{Références à un label du document}

Référence à une Figure associée à un label: Figure \ref{org-brm} de la page \pageref{org-brm}.

\subsection{Références à des adresses}

\subsubsection{Test de href}

Utilisation de href, pour intégrer un lien dans une portion de texte:
\href{https://www.rahombe.com/}{Lien vers une page web}.

\subsubsection{Test de url}

Utilisation de url pour citer un lien cliquable:
\url{https://espantsiranana.mg/actualites}.
