\chapter{Ajout d'un troisième chapitre}

\section{Test de matrice}
    L'équation \ref{matriceA} représente une matrice simplifiée.
 	\begin{equation}
            \label{matriceA}
		[a^m_{\lambda}] = 
		\begin{pmatrix}
    		\mathcal{L}(EU^m_1|SP_1)    &\cdots       & 0 \\
                \vdots                      &\ddots       & \vdots    \\
                0                           &\cdots       & \mathcal{L}(EU^m_{U_m}|SP_{\Lambda})   \\ 
		\end{pmatrix} 
	\end{equation}
    
    L'équation \ref{matriceB} représente une matrice plus complexe.
    
    \begin{equation}
        \label{matriceB}
	\mathcal{A} = 
	\begin{pmatrix}
    	[a^1_1]     &[0]        & [0]       & \cdots           & [0]       & \cdots    & [0] \\ 
		[0]			& [a^1_2] 	& [0] 		& \cdots           & [0]	   &\cdots     & [0] \\
            \vdots  	& \vdots	& \vdots	&  			       & \vdots    &  	       & \vdots	\\ 
		[0] 		& \cdots 	& [a^1_{\lambda}] & \cdots     & [0]       &\cdots     & [0] \\
		\vdots  	& \vdots  	& \vdots  	&[a^m_{\lambda}]   & \vdots    &  		   & \vdots \\
		[0]  		& [0]  		& [0] 		& \cdots 	 	   & [a^M_1]   & \cdots    &[0] \\  
		\vdots  	& \vdots  	& \vdots    & 				   & \vdots    &  \ddots   & \vdots \\ 
 		[0] 			& [0] 		& [0] 		& \cdots 		   & [0] 	   & \cdots    &[a^M_{\Lambda}] \\ 
		\end{pmatrix}
		\qquad b = 
		\begin{pmatrix}
			[\tau_1]\\ 
			[\tau_1]\\ 
			\vdots \\ 
			[\tau_1]\\ 
			\vdots \\ 
			[\tau_m]\\ 
			\vdots \\
			[\tau_M]\\  
		\end{pmatrix}
    \end{equation}

Le tableau \ref{chap3:tableau1} représente un autre tableau.

\begin{table}
    \parbox{0.65\textwidth}{%
    \caption{Un tableau}%
    \label{chap3:tableau1}} % Contrainte manuelle de la largeur de la légende
    \begin{tabular}{|c|c|c|c|c|c|c|c|}
	\hline
		      {\bf titre} & {\bf titre} & {\bf titre} & {\bf titre} & {\bf titre} & {\bf titre} & {\bf titre} & {\bf titre} \\
	  \hline
			blá & blá & blá & blá & blá & blá & blá & blá \\
	  \hline
			blá & blá & blá & blá & blá & blá & blá & blá \\
	  \hline
			blá & blá & blá & blá & blá & blá & blá & blá \\
	  \hline
			blá & blá & blá & blá & blá & blá & blá & blá \\
	  \hline
    \end{tabular}
\end{table}
